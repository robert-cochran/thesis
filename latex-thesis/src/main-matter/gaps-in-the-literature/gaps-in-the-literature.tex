\chapter{Gaps in the Literature}
From the research done nothing currently uses an online based approach to speech classification. The novel entropy approach from [7] is a newly developed method that allows for estimations done in real time using small sample sets. For such a system to work requires speech elements that can be easily digitised and deliver high amounts of meaningful information. Although prosodic based speech classification does exist, it utilises complex techniques that are offline based, meaning calculations require their entire datasets. This method instead uses a subsampling technique to provide accurate results given incomplete information. This paper will investigate the properties of the pause prosodic speech element in determining its effectiveness for an online based speech classification system.


%is able to perform analysis for early diagnosis as fast entropy is a novel tool that hasn’t been produced anywhere else or the process doesn’t have a close mimic?  (do any other of those allow for that? Winterlight?) Ours aims to be mobile, resource cheap, low sample size needed (i.e. fast), easy, reliable, very close to the full method. This means since the other methods require huge sample sizes to perform estimations, this can be done continuously to provide reports of outlier behaviour in real time (why would it be necessary to do these estimations constantly in real time? As in would continual use of classification be necessary for some reason or would it just make a potential diagnosis faster because the results would be appearing in real time rather than a long analysis at the end (which would seem more accurate and thus better for diagnosis honestly)). 
%
%Currently, no research or system has been put into place specifically for looking at how an entropy classification system using fast entropy has been found. Although [2] was extremely related, their methodology was extremely uncontrolled. The results for [2] showed that prosody was not a reliable indicator of trouble in natural, unscripted conversations. However, when using an actor to express more emotion trouble was detected with much more reliability. [2] explain this by saying natural conversations may have all these emotions but not express them as well. It could have been the results were poor because the behaviour in question was not optimized appropriately to the behaviour in question (in their case anger). There is good evidence for this explanation since [2] describes their aim was not to optimize single classes or focus on one specific feature but rather try to show what a successful approach towards model could be taken. The results from this experiment are not taken too seriously as correctness was not shown to be ensured either through understanding the behaviour correctly to model it or to ensure the models were actually detecting what they should have. They also provided a different setting looked at a 
%different conversational behaviour of trouble in communication (anger vs dementia). One important aspect looked at in the paper was the use of controlled tests using actors which was a good choice. 
%For building effective real-time analytical entropy classification systems around Fast Entropy, there must be much more rigour into how the behaviours are chosen, studied and symbolized (including size of symbol set and how clustering was done to form those symbol sets). Pause detection has been done before, as provided good results, it be used to classify? 
%Evaluate what potential acoustic-prosodic events are applicable to developing a real-time behavioural classification need to develop a new alphabet 


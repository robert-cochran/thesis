%%%%%
% DIALECTS
%%%%%
\section{Dialects}
\subsubsection{English dialects}

\paragraph{ca/CallFriend/eng-s}
\subparagraph{Conversation type may change from southern to northern}
eng-s means english-southern

\paragraph{ca/CallFriend/eng-n}
\subparagraph{Conversation type may change from southern to northern}
eng-n means english-northern

%%%%%
% RELATIONSHIP TYPE
%%%%%
\subsubsection{Conversation Partner Relationship}
\paragraph{ca/CallFriend/eng-n + ca/CallFriend/eng-n}
\subparagraph{Reason for experiment: } Calling friends can lead to more natural conversations types. Should think of more reasons here.
For this test northern and southern were grouped together. Unless CallHome was specifically done in one region we can't know which one to study against thus both are used to be safe. NOTE LOOK AND SEE IF CALLHOME DOES SPECIFY WHERE IT IS FOR.

\paragraph{ca/CallHome/eng}
\subparagraph{Reason for Experiment: } Calling home may lead to a different conversation type/style. Calling home can lead to different conversation types than calling a friend. Hopefully this can be seen in the way symbols are distributed when analysed and ranked. This will help in analysing future conversations as it gives insight into how conversation changes through different parties. The variance between different types of natural conversations should be taken into consideration in determining what is a typical conversation and where those bounds start to change. By taking a number of different conversation types it also allows us to build a more robust model that isn't affected solely by someone talking a different party and thus changing demeanour.
URL: http://ca.talkbank.org/access/CallHome/
FILE: ...

Note: Northern vs Southern is not a defined subcategory for calling home. 

%%
% FRIEND
%%
\subsubsection{Language Type}
\paragraph{ca/CallFriend/eng} Both north and south combined again for these experiments
\subparagraph{Reason for experiment:} 

\paragraph{ca/callFriend/jpn}
\subparagraph{Reason for experiment:} Pauses can change from culture to culture (as show in paper??). So one culture may find a certain pause has a specific meaning attached (either by type or length) that other cultures don't. This change in culture can change how pauses are used and thus change the frequency of certain pause lengths occurring. This experiment aims to see if this cultural change can be viewed from the data. \\
URL: https://media.talkbank.org/ca/CallHome/jpn/0wav/
FILE: 0921.wav
Raw pause output
Histogram: 
Findings: Was anything noticed from the raw pause output, its length, the lengths that showed up, the distribution of the lengths, the proportion of lengths, the number of lengths (does pausing just happen less often?)? Similar to how pauses were distributed in monologue 

%%
% HOME
%%
\paragraph{ca/callHome/eng} Both north and south combined again for these experiments
\subparagraph{Reason for experiment:} 

\paragraph{ca/callHome/jpn}
\subparagraph{Reason for experiment:} 

%%
% FRIEND AND HOME
%%
\paragraph{ca/callHome/eng} Both north and south combined again for these experiments
\subparagraph{Reason for experiment:} 

\paragraph{ca/callHome/jpn}
\subparagraph{Reason for experiment:} 



%%%%%%
% EMOTION 
%%%%%%
%% 
% DISTRESS
%%
\paragraph{ca/Examples}
\subparagraph{Reason for experiment:} Analysing a conversation of distress can show a significant conversation type where pausing would shorten. This could be measured to see how it varies to natural conversations between friends or adults. 
URL: https://media.talkbank.org/ca/Examples/
FILE: 911.wav
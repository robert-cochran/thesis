\subsection{Pause Analysis}
%%%%%
% LANGUAGE - JAPAN v ENGLISH
%%%%%

%Although this isn't strictly good practice, combine together the groups that showed no difference intergroup on english and assume the same results would apply for Japanese. Once you get whole results for japanese and english then break down further. Just do this to try and get SOME kind of positive results at least. However, we may be bottlenecked by the pause code analysis and how many symbols we get back ultimately. 
%
%\paragraph{plots}
%Existing histograms (of interviews so not the same) and other graphs:
%A Large-Scale Multilingual Study of Silent Pause Duration by Estelle Campione \& Jean Véronis 
%The Duration of Speech Pauses in a Multilingual Environment by Mike Demol, Werner Verhelst, Piet Verhoeve 
%A Study of Speech Pauses for Multilingual Time-Scaling Applications by Mike Demol, Werner Verhelst, Piet Verhoeve 
%Different Interpretations of Pauses in Natural Conversation ---Japanese, Chinese and Americans by Yuka Shigemitsu*
%
%\paragraph{Work done so far}
%I've created a histogram of all the avg pauses of each audio file. 
%I've created a list of all outliers and done created histograms excluding anything below 200 (maybe 500?) (Should I exclude cases that are statistically out of the 60\% range?  Like below 300? above 1200? removing from below and above are signs that something went wrong in the test)
%So once outliers are removed I should show the before and after of the histogram?
%Histograms of the pause averages for all files, nothing below 200, and nothing below 400 show a clear tendency towards smaller pauses as the outliers are removed. This agrees with the results produced from the high quality abc interviews where the results showed the average pauses to be around xxx length. \\
%This test was used to try and find what pause numbers delivered the most consistent results to itself (so pauses not all over the place on a histogram) but also consistent with previous findings. As the low grade files were removed it became clear that more pauses tended to mean better quality output, but this wasn't always the case. Even some high output files, when examined closely, showed a tendency to over examine and assign pause (need to verify this if I'm going to say it). \\
%
%Which means fewer and fewer audio files that can be used to produce actual reliable results. The original number of files for jpn started at 119, the files that produced a total number of pauses above 200 was 55, and above 400 was 32. Although the database itself provides a large number of audio files, the vast majority are not reliable as a data source to draw decisive results from. Some files produced much higher values (~3500) indicating a need for further examination.These results are in contrast with the abc interview results where variance for those was xxxx, whereas variance for these are xxx. \\
%
%However, since work was already done to create the files, the results produced are given below. These may or may not provide some insight for future work. \\
%
%A best fit line was added to the histograms to help visualise what a normal dist might be over this sample set.
%  
%
%\paragraph{To Do:} 
%increase bin size for jpn_avgs, so many are around 100 that it becomes impossible to 
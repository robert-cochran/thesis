\begin{thebibliography}{99}


\addcontentsline{toc}{chapter}{Bibliography}


\bibitem{lamport} L.~Lamport, \emph{\LaTeX: A Document Preparation
System}, 2nd ed. (Addison-Wesley, 1994).
\bibitem{LABEL2} REFERENCE 2
\bibitem{ETC.} Etc.


\bibliography{src/back-matter/bibliography/bib/bibliography} 
\bibliographystyle{ieeetr}


\end{thebibliography}



[1]


[2]
[Prosody, information structure and evaluation] 

[3]
[Shannon Entropy]

[4]
[The Duration of Speech Pauses in a Multilingual Environment]

[5]
[A Study of Speech Pauses for Multilingual Time-Scaling Applications] 

[6]
[Comparative Study of Classification Algorithms used in Sentiment Analysis]

[7]
[Fast Entropy Estimation for Natural Sequences] 

[8]
[An analysis of trouble and repair in the natural conversations of people with dementia of the Alzheimer’s type]

[9]
[How To Find Trouble In Communication]

[10]
[Communication Breakdown : A Pragmatics Problem]

[11]
[TensorFlow]

[12]
[Speech Feature Extraction and Classification: A Comparative Review]

[13]
[Machine Learning Paradigms for Speech Recognition: An Overview]

[14]
[Real-time Speech and Music Classification by Large Audio Feature Space Extraction by Florian Eyben]

[15]
[Entropy-Driven Dialog for Topic Classification: Detecting and Tackling Uncertainty]

[16]
[Measurement and Classification of Humans and Bots in Internet Chat]

[17]
[On the performance of Fisher Information Measure and Shannon entropy estimators]

[18]
[Exponential Intuitionistic Fuzzy Information Measure with Assessment of Service Quality]

[19]
[Cross-entropy clustering framework for catchment classification]

[20]
[Isolating effects of age with fair representation learning when assessing dementia]

[21]
[Detecting cognitive impairments by agreeing on interpretations of linguistic features]

[22]
[The Duration of Speech Pauses in a Multilingual Environment]

[23]
[Calpy Github]

[24]
[Article 1: Different Interpretations of Pauses in Natural Conversation ---Japanese, Chinese and Americans 1]

[25]
[Article 3: A Large-Scale Multilingual Study of Silent Pause Duration by Campione and Veronis]